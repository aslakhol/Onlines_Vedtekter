\chapter{Økonomi}
\label{chap:okonomi}
\vspace{23pt}

\section{Budsjetterte midler}
Hovedstyret skal forvalte foreningens økonomi etter de vedtak som er fattet av generalforsamlingen og Hovedstyret under fastsettelse av budsjettet. Økonomiansvarlige kan ikke ta avgjørelser rundt fordeling eller forvaltning av budsjetterte midler til ulike prosjekter i regi av linjeforeningen, disse avgjørelsene tas av den spesifikke komiteen eller Hovedstyret.

Ikke-budsjetterte utgifter må godkjennes av Hovedstyret. Refundering av disse utgiftene vil kun forekomme dersom utgiften er godkjent.

\subsection{Fastsettelse av budsjett}
Linjeforeningens budsjett for neste år settes av Bank- og økonomikomiteen på et årlig budsjettmøte i løpet av høstsemesteret. Dette skal deretter godkjennes av Hovedstyret. Budsjettet må være godkjent innen 1. desember for året som følger.

Budsjettet skal være forsvarlig og det skal ikke, med mindre sterke grunner taler for det, budsjetteres med tap eller uten en ansvarlig sikkerhetsmargin.

\subsection{Revidering av budsjett}
Linjeforeningens budsjett for høstsemesteret kan revideres av Bank- og økonomikomiteen i løpet av våren. Revidert budsjett skal godkjennes av Hovedstyret innen 1. mai.

\subsection{Offentliggjøring av budsjett}
Onlines budsjett skal være tilgjengelig for alle linjeforeningens medlemmer.

\section{Fond}
\vspace{23pt}

Onlines formue forvaltes av Onlines Fond. Onlines Fond er definert ved Onlines Fonds vedtekter som eksisterer som vedlegg til Onlines vedtekter.

\section{Overføring av midler}
\vspace{23pt}

Hovedstyret skal under Onlines generalforsamling legge frem forslag til et beløp som skal overføres til fondet.
