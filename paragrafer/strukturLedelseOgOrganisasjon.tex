\chapter{Struktur, ledelse og organisasjon}
\section{Hovedstyret\newline}
Hovedstyret er linjeforeningens høyeste organ mellom generalforsamlingene. Hovedstyrets medlemmer velges på generalforsamlingen og skal drive linjeforeningen mellom generalforsamlingene. For at Hovedstyret skal være beslutningsdyktig må minst halvparten (1/2) av representantene være tilstede.\newline

Ingen kan inneha to verv i Hovedstyret. Leder har dobbeltstemme ved stemmelikhet. Hovedstyrets møter er lukket, men gjester kan inviteres dersom Hovedstyret ønsker dette. Hovedstyret skrives med stor ’H’ på samme måte som egennavn.

Leder av linjeforeningen Online har HMS-ansvar i organisasjonen. 


%----
\subsection{Hovedstyrets sammensetning}{
Hovedstyret består av:
\begin{liste}
	\item Leder
	\item Nestleder
	\item Leder av Arrangementskomiteen
	\item Leder av Bank- og økonomikomiteen
	\item Leder av Bedriftskomiteen
	\item Leder av Drifts- og utviklingskomiteen
	\item Leder av Fag- og kurskomiteen
	\item Leder av Profil- og aviskomiteen
	\item Leder av Trivselskomiteen
\end{liste}
}


\subsection{Fravær av hovedstyremedlem}{
Dersom leder av Hovedstyret blir fraværende er det nestleder i Hovedstyret som skal fylle stillingen. Dersom en komitéleder blir fraværende er det komitélederens nestleder som tar over komitélederens plikter, oppgaver og rettigheter. Dersom en nestleder ikke er tilgjengelig plikter Hovedstyret å fylle stillingen. 

Dersom leder og/eller nestleder av linjeforeningen blir fraværende i den grad at det går utover vervets arbeidsoppgaver skal det innkalles til ekstraordinær generalforsamling for å fylle vervet.
}
\subsection{Krav til kandidater}{
Kandidater til Hovedstyret må ha innehatt et verv i en av kjernekomiteene i linjeforeningen i minst ett (1) semester med unntak av leder av bank- og økonomikomiteen. Om en kandidat ikke har innehatt et verv i en kjernekomité, må kandidaten foreslås av valgkomiteen.
}



%----
\vspace{-23pt}
\section{Kjernekomiteer\newline}
\label{sec:komiteer}

Alle kjernekomiteer består av minimum en leder, nestleder og en økonomiansvarlig. Enhver komité skal utarbeide sine egne retningslinjer som skal legges frem for og godkjennes av Hovedstyret. \newline
							
Kun medlemmer av linjeforeningen kan inneha verv i en kjernekomité. Dersom studenten ikke lengre kvalifiserer til medlemskap i linjeforeningen kan vervet fortsette etter avtale med Hovedstyret. 

%----
\vspace{-10pt}
\subsubsection{Arrangementskomiteen}{
Komiteens hovedoppgave er å koordinere og gjennomføre sosiale arrangement. Komiteens navn forkortes arrkom.

}

%----
\vspace{-10pt}
\subsubsection{Bank- og økonomikomiteen}{
Komiteens hovedoppgave er å administrere linjeforeningens økonomi. Komiteens medlemmer utgjøres av de økonomiansvarlige fra de andre komiteene. Komiteens navn forkortes bankom.

}

%----
\vspace{-10pt}
\subsubsection{Bedriftskomiteen}{
Komiteens hovedoppgaver er å være et bindeledd mellom linjeforeningens medlemmer og næringslivet, og å utarbeide en hovedsamarbeidssamtale med en bedrift for Linjeforeningen Online. Komiteens navn forkortes bedKom.
}

%----
\vspace{-10pt}
\subsubsection{Drifts- og utviklingskomiteen}{
Komiteens hovedoppgave er å utvikle og vedlikeholde linjeforeningens datasystemer. Komiteens navn forkortes dotkom.

}

%----
\vspace{-10pt}
\subsubsection{Fag- og kurskomiteen}{
Komiteens hovedoppgave er å koordinere og gjennomføre arrangement som tilbyr faglig innhold, primært for linjeforeningens egne medlemmer. Komiteens navn forkortes fagkom.
}

%----
\vspace{-10pt}
\subsubsection{Profil- og aviskomiteen}{
Komiteens hovedoppgave er å sikre kvalitet på profileringsmateriell, samt gi ut linjeforeningens tidsskrift. Komiteens navn forkortes prokom.
}

%----
\vspace{-10pt}
\subsubsection{Trivselskomiteen}{
Komiteens hovedoppgave er å sørge for økt trivsel blant informatikere i hverdagen og har hovedansvar for linjeforeningskontoret. Komiteens navn forkortes trikom.
}

%----
\vspace{-10pt}
\subsubsection{Seniorkomiteen}{
Komiteens hovedoppgave vil være å bistå med kunnskap, erfaring og assistanse i linjeforeningens daglige drift. Komiteens navn forkortes seniorkom. For å søke seg til seniorkomiteen må man ha hatt et aktivt verv i linjforeningen i fire (4) semester.					

Seniorkomiteen velger selv sin leder. Leder av Seniorkomiteen har møte- og talerett i Hovedstyret.
}

%----
\vspace{-10pt}
\section{Nodekomiteer\newline}{
En nodekomite er underlagt en av komiteene beskrevet i 4.2, eller direkte underlagt Hovedstyret, og plikter å holde den ansvarlige kjernekomiteen løpende oppdatert på sitt virke. Nodekomiteer som har eksistert i to år eller mer kan bare offisielt nedlegges av Generalforsamlingen.
}



%----
\subsection{Jubileumskomiteen}{
Komiteens hovedoppgave er å organisere arrangement i forbindelse med linjeforeningens jubileer. Komiteens navn forkortes jubkom.

}

%----
\subsection{Velkomstkomiteen}{
Komiteens hovedoppgave er å organisere fadderperiode for nye studenter som oppfyller de krav for medlemskap som er listet under 5.1. Komiteens navn forkortes velkom.
}

\subsection{Ekskursjonskomiteen}{
Komiteens hovedoppgave er å organisere hovedekskursjonen. Komiteens navn forkortes ekskom.
}
%----
\subsection{Redaksjonen}{
Komiteens hovedoppgave er å gi ut linjeforeningens avis. Redaktøren står fritt fra linjeforeningen, men er underlagt de retningslinjer og avtaler som finnes mellom redaksjonen og linjeforeningen. Redaktøren velger selv redaksjonsmedlemmer, også blant personer utenfor linjeforeningen. Redaksjonsmedlemmer som ikke innfrir krav til medlemskap i linjeforeningen, som definert under kapittel 5, er ikke medlemmer av linjeforeningen.
}

\subsection{Informatikernes IT-ekskursjon}{
Komiteens hovedoppgave er å arrangere ekskursjon til Oslo for masterstudenter hver høst. Komiteens navn forkortes Itex.
}

\subsection{Applikasjonskomiteen}{
Komiteens hovedoppgave er å utvikle og drifte egne it-systemer. Komiteens navn forkortes appkom.
}

\subsection{Casual gaming}{
Komiteens hovedoppgave er å organisere LAN. Casual Gaming opererer frittstående fra linjeforeningen.
}


\section{Interessegrupper\newline}{
Interessegrupper kan opprettes av Online-medlemmer som ønsker å dekke et behov som gagner informatikkstudenter. Disse grupperingene formulerer sine egne retningslinjer, og budsjett ved behov, som godkjennes av Hovedstyret. Hovedstyret avgjør også hvorvidt gruppen underlegges en eksisterende komité, eller Hovedstyret selv. 
}

\section{Permisjon eller oppsigelse fra komité\newline} 

\subsection{Pause fra sitt engasjement}{
Ved permisjon fra en komité er man fullstendig fritatt de pliktene komitevervet medførte. 

Et komitémedlem kan søke om permisjon fra et komitéverv i Online. Man må ha sittet i en komité i minst ett (1) semester for å kunne søke permisjon. Dersom permisjonen varer lengre enn to (2) semestere vil medlemmets verv opphøre.
}

\subsection{Verv i Hovedstyret}{
Dersom et komitémedlem blir valgt til et av følgende hovedstyreverv vil medlemmet automatisk få permisjon fra sin komité, og kan fritt \linebreak returnere til denne ved endt engasjement i Hovedstyret:
\begin{liste}
	\item Leder
	\item Nestleder
	\item Leder for banKom
\end{liste}
}

\subsection{Advarsel og oppsigelse}{
Leder av en komité har rett til å si opp et medlem av sin egen komité. Oppsigelse skal kun finne sted i tilfeller der det blir ansett som høyst nødvendig for å beskytte komiteens samhold, initiativ, integritet eller profesjonalitet. Leder av komiteen plikter å konsultere leder av linjeforeningen i forkant av en eventuell advarsel eller oppsigelse.
}

\section{Mislighold av verv\newline}{
Om et komitémedlem eller en innehaver av linjeforeningsverv misligholder sine arbeidsoppgaver, kan ethvert medlem av linjeforeningen stille mistillitsforslag overfor vedkommende. Mistillitsforslaget skal leveres skriftlig til Hovedstyret, som skal behandle saken. Ved mistillitsforslag mot et hovedstyremedlem blir den anklagede suspendert inntil Hovedstyret har kommet med en avgjørelse. Mistillitsforslaget leses opp i Hovedstyret, deretter skal den anklagede få en mulighet til å forsvare seg før Hovedstyret diskuterer og avgjør saken uten den anklagede til stede. Dersom det stilles mistillitsforslag til flere styremedlemmer av gangen, skal disse behandles ved ekstraordinær generalforsamling. 
}

\section{Vervvarighet}{
Et verv i linjeforeningen varer i seks (6) semestre fra måneden man ble tatt opp. Dersom man ønsker å være et aktivt komitemedlem etter disse seks (6) semestrene kan man søke til Hovedstyret om forlengelse av Online-vervet for to semestre av gangen. Alle verv i kjernekomiteene i linjeforeningen teller på de seks vervsemestrene, inkludert verv i Hovedstyret, men ekskludert verv i seniorkomiteen. 
}

\section{Æresmedlemmer\newline}{
Æresmedlemmer er personer som har bidratt i særskilt stor grad til å avansere linjeforeningen, informatikkfaget eller bidratt i stor grad til saker hvor linjeforeningen eller informatikk er påvirket.
Æresmedlem er en tittel hovedsaklig for personer som ikke har vært medlem av linjeforeningen.
Æresmedlemmer utnevnes av Hovedstyret.

}
\section{Ridderne av det indre lager}

Linjeforeningen har en Ridderorden for medlemmer som gjennom sitt arbeid har utmerket seg. Denne ordenen er selvorganisert, og faller utenfor daglig drift av linjeforeningen.

\subsection{Formål}
Ridderordenen er også tilgjengelig som en kilde for kunnskap, historie og meninger. Et mål for Ridderordenen er å kunne bistå med tanker rundt organisasjonen, verdiene til linjeforeningen og langsiktige planer.
Ridderordenens plikt er å etterstrebe en god tilstedeværelse under opptaket av nye medlemmer til linjeforeningen. Ansamlingen Riddere i denne anledningen betegnes som Eldsterådet.


\subsection{Organisering}{
Ridderordenen består av flere grader hvor høyere grader betegner større engasjement.
}

\subsection{Medlemsskap}{
Ridderordenen bestyrer selv sitt eget opptak og vurdering av kandidater. Utnevnelser foregår i formelle anledninger der en betydelig del av linjeforeningens medlemmer er samlet. Før Ridderordenen kan utnevne kandidater skal sittende leder av Hovedstyret underrettes om hvilke kandidater det gjelder. For å vurderes til utnevnelse må en kandidat på et tidspunkt ha oppfylt kravene til medlemskap, jfr. 5. Engasjement som vektlegges når Ridderordenen vurderer kandidater inkluderer verv i linjeforeningen og andre organisasjoner som er direkte knyttet til linjeforeningen.
}
